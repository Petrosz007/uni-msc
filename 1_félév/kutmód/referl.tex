\documentclass[compress]{beamer}
\mode<presentation>
{
  \usetheme{Warsaw}
}

\setbeamertemplate{navigation symbols}{}

\usepackage[utf8]{inputenc}
\usepackage[T1]{fontenc}
\usepackage{alltt}
\usepackage{tikz}


\usetikzlibrary{shapes.geometric}


%%%%%%%%%%%%%%%%%%%%%%%%%%%%%%%%%%%%%
%
% Matematikai jelek es roviditesek:
%
%%%%%%%%%%%%%%%%%%%%%%%%%%%%%%%%%%%%%
\def\u{\hfill\break}
%\def\unless {\triangleright}
\def\unless {\rhd}
\def\ensures {\mapsto}
%\def\leadsto {\hookrightarrow}
\def\inev {\hookrightarrow}
\def\lto{{\hookrightarrow\hspace*{-2 ex}\circ\hspace*{1 ex}}}
\def\eto{{\mapsto\hspace*{-2 ex}\circ\hspace*{1 ex}}}
%\def \leadsto van
\def\initially {{\rm INIT}}
\def\fixpont {{\rm FP}}
\def\fps{{fixpont}_S}
\def\term {{\rm TERM}}
\def\doboz{{$\sqcap$}\llap{$\sqcup$}}
\def\inv{{\rm inv}}
\def\INV{{\rm INV}}
%\def\kons{{\rm cons}}
\def\true{\uparrow}
\def\false{\downarrow}
\def\init{{\rm INIT}}
\def\underset#1#2#3{{ {\displaystyle #3}\atop {#1} }}
\def\Ordo{O} % hz
\def\ih#1{{\lceil#1\rceil}}
\def\Di#1{{[#1]}}
\def\parcfv{\longrightarrow}
\def\fv{\longmapsto}
%\def\ha-akkor{\Rightarrow}
%\def\impl{\rightarrow}
%\def\acsa{\Longleftrightarrow}
%\def\akkor-ha{\Longleftarrow}
\def\eqdef{{::=}}
\def\legy{{::=}}
%\def\eqdef{\;\mathrel{\displaystyle{\mathop=^{\rm{\fiverm def}}}}\;}
\def\underset#1#2#3{{ {\displaystyle #3}\atop {#1} }}


\title[Teaching Principles  of Dependable Distributed Software]{Teaching Principles of Dependable Distributed Software}

\author[Zoltán Horváth, \texttt{hz@inf.elte.hu}]{Zoltán Horváth \\ 
\texttt{hz@inf.elte.hu}}
\institute{{\small Faculty of Informatics, Eötvös Loránd University, Budapest}}
\date{14th Workshop on Software Engineering Education and Reverse Engineering \\ Sinaia, Aug 26, 2014}


\begin{document}

\maketitle

\section{Dependable Software}

\begin{frame}{Software – what our everyday life is depending on}

\begin{columns}
\column{7cm}{}

\begin{itemize}
\item communication: phone, e-mail, social media
\item navigation
\item home banking
\item healthcare administration, medical devices 
\item car -- in-vechile software: diagnostic, safety, driver assistance
\item aircraft --  autopilot control system ("Who's really flying the plane?")
\item nuclear power plant  control system safety software
\end{itemize}

\column{5cm}{}
\begin{figure}
\includegraphics%[width=1.0\textwidth, totalheight=1.0\textheight]
{atmswitch.jpg}
\end{figure}
\end{columns}

\end{frame}

\begin{frame}{Infamous software bugs}
\begin{itemize}

\item The Mars Climate Orbiter doesn't orbit (1998, metric system, \$327.6 million)
\end{itemize}
 \begin{figure}
\includegraphics[width=0.4\textwidth, totalheight=0.4\textheight]
 {CM-MarsCO-pic-array.jpg}
\end{figure}
\end{frame}


\begin{frame}{}
\begin{itemize}

\item Call waiting ... and waiting ... and waiting (On Jan. 15, 1990, around 60,000 AT\&T custumers, congestion lead to cascade reset of 114 switches)

\item Therac-25 Medical Accelerator disaster (1985, race condition, two modes)

\item Soviet early-warning system (Sep. 23, 1983, Petrov)

\end{itemize}
\begin{figure}
\includegraphics[width=0.5\textwidth,totalheight=0.5\textheight]{soviet.jpg}
\end{figure}
\end{frame}


\begin{frame}{}
\begin{itemize}

\item Nuclear Power Plant shutdown (June 5, 2008, USA, software update in a distributed system)

\end{itemize}

\begin{figure}
\includegraphics[width=0.5\textwidth, totalheight=0.5\textheight]{plant-hatch.jpg}
\end{figure}

\begin{itemize}

\item Ford in-vechile software failiure - hw. reset at Körösf\H o (2014)

\item Samsung mobile network connection error (2014, flight mode off and on) 

\end{itemize}

\end{frame}

\begin{frame}{Correctness of Distributed programs}

%\centerline{}
\includegraphics[scale=0.7]{fa.pdf}

\begin{itemize}

\item interleaving, branching time semantics of distributed and parallel programs

\item testing of properties, reachable states

\item specification of the problem to solve

\item static verification, analysis, calculation of reachable states

\end{itemize}

\end{frame}




\begin{frame}{Always true is not always invariant}

\begin{itemize}

\item  P  invariant -- P holds initialy and P is preserved (also in unreachable states)

\item  P2  always true  -- P2 holds in all reachable states 

\end{itemize}

\includegraphics[scale=0.7]{inv.pdf}

Only invariants do compose!

\end{frame}       

\begin{frame}{}

Motivation for using formal methods: 

- safety critical applications

- safe application of software components

- {\bf primary goal}: sound concepts about distributed and parallel 
programs

\end{frame}




\begin{frame}{Properties of the formal model}


 We need  a  formal model, which is appropriate for 
specification of {\em problems\/} and developing the {\em  solutions}
of problems in case of  
{\em parallel and distributed systems}. 


The introduced model

\begin{itemize}


\item  is an extension of a relational model of nondeterministic 
sequential programs,

 
\item provides tools for stepwise refinement of
problems, in a functional approach,

 
 
\item uses the concept of iterative abstract program
of UNITY,



\item the concept of solution is based on the comparison 
of the  problem as a relation and the behaviour relation of the
program.


\end{itemize}  

\end{frame}

\begin{frame}{Example problem: the dining philosophers}


\centerline{\includegraphics[scale=0.5]{filo.pdf}}

States: thinking:t, forks in hands:f, eating:e, at home:h,  

\end{frame}

\begin{frame}{}


Some requirements (problem specification):  $\forall  i : $

unless: $ f(i).t \unless f(i).f \lor f(i).h$

unless: $ f(i).f \unless f(i).e$
 
ensures: $f(i).e \ensures f(i).t$

inevitable leads-to: $f(i).t \inev f(i).h$

invariant: $ (f(i).e \rightarrow (\lnot f(i+1).e \land f(i-1).e))  \in 
\inv$

fixed point: $ \fixpont \Rightarrow f(i).h$

termination: $ \forall i: f(i).t \in \term $

\end{frame}

\begin{frame}{}

\centerline{\bf Program, solution}

$S = (SKIP, \u
\{
\underset{i \in [1..n-1]}\to{\Box}  a(i),a(i+1) := a(i+1),a(i), \  \mbox{if} \ 
a(i) > a(i+1) \})$

\bigskip

Abstract execution model: No control flow, free processors select 
assignments asynchronously

Program: scheduling, processes, location, communication 
infrastructure, language 

Example: C++/PVM PC-cluster  (Parallel Virtual Machine) / Erlang

Solution: Specification requirements are satisfied by
program properties

\end{frame}



\begin{frame}{}

\begin{itemize}


\item The notion of the state space makes
it possible to define the semantical meaning of a
problem independently of any program.\

\item The generalized
concept of a problem is applicable for cases in which termination is
not required but the behaviour of the specified system is
restricted by safety and progress properties.\

\item The solution of a
problem may be a sequential program, a parallel one, or even {\em a
program built up from both sequential and parallel components}.\

\end{itemize}


\end{frame}




 
 
\begin{frame}{Parallelism and Functional Programming}

\begin{itemize}
 
\item DClean - a coordination language for type safe distributed cooperation of Clean programs - with Viktória Zsók, Zoltán Hernyák

\item Sparkle-T - proof tool for temporal properties (e.g. invariants) of Clean programs - with Máté Tejfel and Tamás Kozsik

\item Static analysis - to support code comprehension at industrial level in Erlang - with Melinda Tóth, István Bozó and others

\item Paraphrase  - property preserving transformations of Erlang programs to enable parallel multicore execution - with Tamás Kozsik, Melinda Tóth, István Bozó, Judit Kőszegi, Dániel Horpácsi, Viktória Fördős and others
  
\end{itemize}
\end{frame} 


\begin{frame}{Literature}


- Chandy, K.M., Misra, J.: {\em Parallel Program Design - A Foundation}.
Addison-Wesley, 1989.

- Misra, J.: {\em A Discipline of Multiprogramming - Programming Theory 
for Distributed Applications}. Springer, 2001.

- Horv\'ath Z.:  Parallel asynchronous computation 
of the values of
 an associative function.\ {\em Acta Cybernetica}, Vol.\ 12, No. 1,
 Szeged (1995) 83-94.

- Horv\'ath Z.: The Formal Specification of a 
Problem Solved by a Parallel Program -- a Relational Model.\

\end{frame}


\end{document}
